\documentclass[12pt, a4paper]{article}

% Pakete
\usepackage[utf8]{inputenc}
\usepackage[T1]{fontenc}
\usepackage[ngerman]{babel}
\usepackage{helvet} % Schriftart "Arial" 
\usepackage{setspace} % 1,5-zeiliger Zeilenabstand
\usepackage{titlesec} % Für Anpassungen an den Überschriften
\usepackage{geometry} % Für Anpassungen an den Rändern
\usepackage{hyperref} % Für Verlinkungen im Dokument
\usepackage{tocloft} % Für Anpassungen an das Inhaltsverzeichnis
\usepackage{apacite} % Für Zitate im APA-Stil

% Formatierung
\renewcommand{\familydefault}{\sfdefault} % Schriftart "Arial"
\geometry{left=3cm,right=3cm,top=2.5cm,bottom=2.5cm} % Ränder
\setlength{\parskip}{0.5em} % Abstand zwischen Absätzen
\setstretch{1.5} % 1,5-zeiliger Zeilenabstand
\setcounter{tocdepth}{3} % Inhaltsverzeichnis bis Ebene 3
\renewcommand{\cftsecleader}{\cftdotfill{\cftdotsep}} % Punkte im Inhaltsverzeichnis
\hypersetup{ % Einstellungen für Verlinkungen
    colorlinks=true,
    linkcolor=black,
    filecolor=magenta,      
    urlcolor=cyan,
}
\titleformat{\section} % Anpassungen an Überschriftenebene 1
    {\fontsize{14}{16}\bfseries}{\thesection}{0.5em}{}
\titleformat{\subsection} % Anpassungen an Überschriftenebene 2
    {\fontsize{12}{14}\bfseries}{\thesubsection}{0.5em}{}
\titleformat{\subsubsection} % Anpassungen an Überschriftenebene 3
    {\fontsize{12}{14}\itshape}{\thesubsubsection}{0.5em}{}

% Dokumentenanfang
\begin{document}

% Titelseite
\begin{titlepage}
    \begin{center}
        \vspace*{1cm}
        
        \textbf{\LARGE{Prognose über die Stromerzeugung einer Photovoltaikanlage mittels maschinellem Lernen}}
        
        \vspace{0.5cm}
        \Large{Bachelorarbeit}
        
        \vspace{1.5cm}
        \Large{Vorgelegt von}
        
        \vspace{0.5cm}
        \Large{Leopold Schmid}
        
        \vspace{1.5cm}
        \Large{Matrikelnummer:79776}
        
        \vfill
        
        \Large{Fakultät für Elektronik und Informatik}
        
        \vspace{0.5cm}
        \Large{Hochschule Aalen}
        
        \vspace{1.5cm}
        \Large{Datum}
        
    \end{center}
\end{titlepage}


\newpage

\section*{Erklärung}


\newpage


\section*{Verzeichnis häufig verwendeter Symbole und Abkürzungen}

\newpage

\tableofcontents


\newpage
\setcounter{section}{0}


\section{Einführung}

% https://deliverypdf.ssrn.com/delivery.php?ID=741072022102025091073125113103087026048056000029024069069006024045033044114095010039039074015127039038088084074028113120004096071122104086004095015124005018003010012023052041098045084086104027117119025110095086024026081088092106122086093064003097016028011065000109108116004064022082122002124&EXT=pdf&INDEX=TRUE

In der größten bisher durchgeführten Studie zu Klimaangst von jungen Menschen behaupten 45\% der Befragten, dass ihre Gefühle bezüglich des Klimawandels ihr tägliches Leben negativ beeinflussen. Ohne jede Zweifel stellt der Klimawandel uns und die uns folgenden Generationen vor eine nicht zu unterschätzende Herausforderung. Angefangen mit der Tatsache, dass die alleinige Diskussion darüber nicht selten zu einer Spaltung und Polarisierung der Gesellschaft führt. Folglich leitet die hitzige und emotionale Debatte dazu, dass die gegensätzlichen Lager sich immer weiter voneinander entfernen und somit jegliche Grundlage für eine zielführende Diskussion entreißen.

Auf der einen Seite werden die Befürchtungen der anderen für übertrieben, unwichtig und paranoid gehalten. Wissenschaftliche Unsicherheiten werden verwendet, um gesamte Ergebnisse von Studien als unzuverlässig einzustufen. Modelle werden in der Wissenschaft oft vereinfacht, denn die Isolation und Fokussierung auf bestimmte Aspekte kann helfen, um ein besseres Verständnis von Phänomenen in der Natur zu erlangen. Auf Grund von dieser Vereinfachung werden die Klimamodelle als zu ausdruckslos betitelt, um die Komplexität des Klimas von unserem Planeten widerzuspiegeln. Dementsprechend seien die Schlussfolgerungen aus den Studien unzutreffend. Investitionen mehrerer Milliardenbeträge seien nicht gerechtfertigt und politische Vorgaben schaden dem eigenem Land mehr, als dass sie dem Planeten helfen würden.

Am gegenüberliegenden Ufer wird gemahnt, die Ernsthaftigkeit der Situation nicht zu unterschätzen. Gewarnt wird, dass die Konsequenzen des Klimawandels irreversibel seien, weswegen Treibhausgase unverzüglich auf ein Minimum reduziert werden sollten. Um Folgen wie das Abschmelzen der Eisschilde und Gletscher, das Aussterben verschiedenster Tierarten und der Anstieg des Meeresspiegels zu verhindern, spielen erneuerbare Energien eine entscheidende Rolle. Im Gegensatz zu fossilen Energieträgern erzeugen die Erneuerbaren keine oder nur kaum Treibhausgasemissionen. Bis 2030 sollen die regenerativen Energien 80\% des Strombedarfs Deutschlands decken. Da die Möglichkeiten der Energieerzeugung in Deutschland durch Wasserkraft und Biomasseverbrennung schon heute nahezu ausgeschöpft sind, bilden Solar- und Windenergie einen wichtigen Grundpfeiler um die Ausbauziele zu erreichen. 

Ebenso komplex wie die Klimamodelle, die die Verstrickungen verschiedenster Phänomene in unserer Natur berücksichtigen sollen, ist allerdings die Transformation unseres Stromnetzes. In einem Netz, welches dafür ausgelegt wurde, dass wenige größere Kraftwerke Strom einspeisen, werden künftig immer mehr kleinere, dezentrale Kraftwerke mitwirken. Als Folge kommen diverse Herausforderungen auf uns zu. Unter anderem verträgt nicht jedes elektronische Bauteil in einem Niederspannungsnetz Rücklaufstrom. (Quelle?) Allerdings könnte es genau zu einem solchen Rücklaufstrom kommen, wenn in dem Niederspannungsnetz mehr Strom erzeugt als verbraucht wird. Des Weiteren reagiert ein Stromnetz äußerst empfindlich auf Spannungsschwankungen. Auf Grund dessen müssen Netzbetreiber darauf achten, dass genauso viel Strom verbraucht wie erzeugt wird. Solar und Windenergie zählen zu den fluktuierenden Energieerzeugern, ergo ist die Stromproduktion nur sehr begrenzt regulierbar und somit nicht an die aktuelle Marktnachfrage anpassbar.

Die Relevanz der Abschlussarbeit basiert auf der Annahme, dass eine genauere Prognose über die erzeugte Strommenge helfen würde das Stromnetz zu stabilisieren und Netzschwankungen, ergo das Risiko eines Brown- oder sogar Black-Outs zu minimieren. Anhand der neu gewonnenen
Information könnten kurzfristige Stromimporte beziehungsweise -exporte besser reguliert werden. Ebenso wäre es denkbar, dass man für die Industrie und/ oder Endverbraucher mehr Anreize schafft ihren Stromverbrauch zu einem gewissen Grad an die Verfügbarkeit des Stroms
anzupassen, wobei die Information über die Verfügbarkeit von Strom gleichfalls hilfreich wäre.

% https://zeitung.faz.net/fas/technik-und-motor/2023-05-14/fuenf-am-tag/892835.html

\section{Stand der Technik}

\subsection{Strombedarf in Deutschland}

\subsection{Schwankungen im Stromnetz}

Dementsprechend ist es für uns unverzichtbar, die Wichtigkeit einer stabilen \linebreak Stromerzeugung richtig einzuschätzen und anhand dessen Maßnahmen zu ergreifen, um Spannungsschwankungen im Netz so gut wie möglich zu reduzieren.

\subsection{Maschinelles Lernen}

\newpage

\section{Eigene Fragestellung und methodisches Vorgehen}

Um die zukünftigen Herausforderungen der Energieversorgung zu bewältigen, befasst sich diese Arbeit damit den fluktuierenden Stromerzeuger, die Photovoltaikanlage, so gut wie möglich in unser bestehendes Stromnetz zu integrieren. Die zentrale Fragestellung lautet hierbei, wie man präzise den erzeugten Solarstrom prognostizieren kann. Um dieses Ziel zu erreichen, sollen zuerst entscheidende Parameter ermittelt werden, die die Stromerzeugung einer Solaranlage beeinflussen. Allerdings liegt der Fokus ebenfalls darauf, dass die Berechnung der Prognose für beliebige Photovoltaikanlagen anwendbar ist, indem nur wenige spezifische Informationen über die Anlage in die Berechnung einfließen. Selbstverständlich spielen verschiedenste Eigenschaften der Solaranlage eine entscheidende Rolle, jedoch werden diese im Datenmodell nicht berücksichtigt. Zunehmende Rechenkapazitäten und kostengünstigere Speichermöglichkeiten ermöglichen es, dass für jede einzelne Solaranlage ein eigenes Datenmodell angelegt wird. Somit können die Gegebenheiten der Photovoltaikanlage, wie Quantität und Qualität der Solarpanele, vernachlässigt werden. Ein weiterer Vorteil ist, dass dadurch schwer zu beschaffende Datensätze nicht weiter benötigt werden. Stattdessen wird das Modell fast ausschließlich von Wetterdaten trainiert, welche von verschiedensten Institutionen flächenmäßig und umfangreich aufgezeichnet werden. Die für das Trainieren des maschinellen Lernen Modells benötigten Wetterdaten werden von der Website \textit{https://www.visualcrossing.com} extrahiert. Die Website bietet einen kostenfreien Zugang zu detaillierten historischen Wetterdaten. 

WO SOLARDATEN?

Durch die Literaturrecherche sind bereits die relevanten Faktoren eingegrenzt wurden, allerdings ist exaktere Analyse der Daten unumgänglich. Ebenso gilt es zu überprüfen, ob sich die verschiedenen Merkmale auf unterschiedliche Solaranlagen gleich auswirken. 

Maschinelles Lernen ist eine Technologie, Muster und Zusammenhänge in großen Datenmenge zu verstehen. 

Um die Schwierigkeiten bei der dezentralen Einspeisung von Solarstrom lösen zu können, soll maschinelles Lernen Abhilfe schaffen. Die Technologie verspricht, dass 

\newpage

\section{Wichtige Faktoren bei der Stromerzeugung durch Solarenergie}

Bevor der Stromertag einer Photovoltaikanlage mit Hilfe von künstlicher Intelligenz berechnet werden soll, ist es von immenser Bedeutung die maßgeblichen Faktoren herauszufinden. Wenn ein Modell eingelernt werden soll, können zu viele Faktoren zum \textit{Fluch der Dimensionen} führen. Dieser besagt, dass zu viele Merkmale dazu verleiten Muster und Strukturen sich schwerer erkennen lassen. Die Datenpunkte sind durch die Größe des mehrdimensionalen Raums weiter voneinander entfernt, wodurch die Interpretierbarkeit komplexer wird.

% https://www.visualcrossing.com/resources/documentation/weather-data/how-to-obtain-solar-radiation-data/

\subsection{Wetter-unabhängige Faktoren}


\subsection{Wetter-abhängige Faktoren}


\newpage

\section{Zusammenhänge der Merkmale}

\newpage

\section{Datenvorverarbeitung}


\newpage

\section{Maschinelles Lernen - Auswahl des Modells}



\end{document}


